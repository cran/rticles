%%%%%%%%%%%%%%%%%%%%%%%%%%%%%%%%%%%%%%%%%%%%%%%%%%
%  JASA LaTeX Template File
%  To make articles using JASA.cls, Version 1.1
%  September 14, 2019
%%%%%%%%%%%%%%%%%%%%%%%%%%%%%%%%%%%%%%%%%%%%%%%%%%

%% Step 1:
%% Uncomment the style that you want to use:

%%%%%%% For Preprint
%% For manuscript, 12pt, one column style

\documentclass[$for(classoption)$$classoption$$sep$,$endfor$]{JASA}

%%%%% Preprint Options %%%%%
%% The track changes option allows you to mark changes
%% and will produce a list of changes, their line number
%% and page number at the end of the article.
%\documentclass[preprint,trackchanges]{JASA}


%% NumberedRefs is used for numbered bibliography and citations.
%% Default is Author-Year style.
%% \documentclass[preprint,NumberedRefs]{JASA}

%%%%%%% For Reprint
%% For appearance of finished article; 2 columns, 10 pt fonts

% \documentclass[reprint]{JASA}

%%%%% Reprint Options %%%%%

%% For testing to see if author has exceeded page length request, use 12pt option
%\documentclass[reprint,12pt]{JASA}


%% NumberedRefs is used for numbered bibliography and citations.
%% Default is Author-Year style.
% \documentclass[reprint,NumberedRefs]{JASA}

%% TurnOnLineNumbers
%% Make lines be numbered in reprint style:
% \documentclass[reprint,TurnOnLineNumbers]{JASA}

$if(natbib)$
\usepackage{natbib}
$endif$

$if(listings)$
\usepackage{listings}
$endif$
$if(lhs)$
\lstnewenvironment{code}{\lstset{language=Haskell,basicstyle=\small\ttfamily}}{}
$endif$

% Pandoc syntax highlighting
$if(highlighting-macros)$
$highlighting-macros$
$endif$

% Pandoc citation processing
$if(csl-refs)$
\newlength{\csllabelwidth}
\setlength{\csllabelwidth}{3em}
\newlength{\cslhangindent}
\setlength{\cslhangindent}{1.5em}
% for Pandoc 2.8 to 2.10.1
\newenvironment{cslreferences}%
 {$if(csl-hanging-indent)$\setlength{\parindent}{0pt}%
 \everypar{\setlength{\hangindent}{\cslhangindent}}\ignorespaces$endif$}%
 {\par}
% For Pandoc 2.11+
\newenvironment{CSLReferences}[2] % #1 hanging-ident, #2 entry spacing
{% don't indent paragraphs
 \setlength{\parindent}{0pt}
 % turn on hanging indent if param 1 is 1
 \ifodd #1 \everypar{\setlength{\hangindent}{\cslhangindent}}\ignorespaces\fi
 % set entry spacing
 \ifnum #2 > 0
 \setlength{\parskip}{#2\baselineskip}
 \fi
}%
{}
\usepackage{calc} % for calculating minipage widths
\newcommand{\CSLBlock}[1]{#1\hfill\break}
\newcommand{\CSLLeftMargin}[1]{\parbox[t]{\csllabelwidth}{#1}}
\newcommand{\CSLRightInline}[1]{\parbox[t]{\linewidth - \csllabelwidth}{#1}\break}
\newcommand{\CSLIndent}[1]{\hspace{\cslhangindent}#1}
$endif$

$if(verbatim-in-note)$
\usepackage{fancyvrb}
\VerbatimFootnotes % allows verbatim text in footnotes
$endif$

$for(header-includes)$
$header-includes$
$endfor$

\begin{document}
%% the square bracket argument will send term to running head in
%% preprint, or running foot in reprint style.

\title[$if(shorttitle)$$shorttitle$$endif$]{$title$}

% ie
%\title[JASA/Sample JASA Article]{Sample JASA Article}

%% repeat as needed

$for(author)$
\author{$author.name$}
% ie
%\affiliation{Department1,  University1, City, State ZipCode, Country}
\affiliation{$author.affiliation$}
%% for corresponding author
$if(author.email)$\email{$author.email$}$endif$
%% for additional information
$if(author.thanks)$\thanks{$author.thanks$}$endif$
$endfor$

% ie
% \author{Author Four}
% \email{author.four@university.edu}
% \thanks{Also at Another University, City, State ZipCode, Country.}

%% For preprint only,
%  optional, if you want want this message to appear in upper left corner of title page
$if(preprint_notice)$\preprint{$preprint_notice$}$endif$

%ie
%\preprint{Author, JASA}

% optional, if desired:
%\date{\today}
$if(date)$\date{\today}$endif$

$if(abstract)$
\begin{abstract}
% Put your abstract here. Abstracts are limited to 200 words for
% regular articles and 100 words for Letters to the Editor. Please no
% personal pronouns, also please do not use the words ``new'' and/or
% ``novel'' in the abstract. An article usually includes an abstract, a
% concise summary of the work covered at length in the main body of the
% article.
$abstract$
\end{abstract}
$endif$

%% pacs numbers not used

\maketitle

%  End of title page for Preprint option --------------------------------- %

%% See preprint.tex/.pdf or reprint.tex/.pdf for many examples

$for(include-before)$
$include-before$
$endfor$

%  Body of the article
$body$

% -------------------------------------------------------------------------------------------------------------------
%   Appendix  (optional)

%\appendix
%\section{Appendix title}

%If only one appendix, please use
%\appendix*
%\section{Appendix title}


%=======================================================

%Use \bibliography{<name of your .bib file>}+
%to make your bibliography with BibTeX.

%=======================================================

$if(natbib)$
$if(bibliography)$
$if(biblio-title)$
$if(book-class)$
\renewcommand\bibname{$biblio-title$}
$else$
\renewcommand\refname{$biblio-title$}
$endif$
$endif$
\bibliography{$for(bibliography)$$bibliography$$sep$,$endfor$}
$endif$
$endif$
$if(biblatex)$
\printbibliography$if(biblio-title)$[title=$biblio-title$]$endif$
$endif$

$for(include-after)$
$include-after$
$endfor$

\end{document}
